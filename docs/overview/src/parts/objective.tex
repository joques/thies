\section{Objectives}
\label{sec:objectives}

The main objective of this research project is to build a multi-agent-powered model for waste sorting in public buildings. The study will be guided by the following specific research objectives (RO):

\begin{itemize}
  \item to investigate the data requirements needed for building a multi-agent-powered model for waste sorting at the public refuse;
  \item  to develop a multi-agent-powered model for waste sorting at the public refuse;
  \item validate the developed model.
\end{itemize}

\subsection{MAS Requirements}
\label{sub:obj-requirements}

The first sub-objective focuses on investigating the data requirements needed for building a multi-agent-powered model for waste sorting at the public refuse”. To achieve this objective, we will initially gather data through field surveys, social questionnaires, and face-to-face interviews with residents, recycling sites, and government statistical records in Namibia. We will focus on the characteristics, parameters, decision-making processes, and data relevant to each type of agent in the proposed model. In order to train the model to accurately identify and classify different types of waste using computer vision techniques, we primarily need image data of various waste items, carefully labeled with their corresponding categories (such as plastic, paper, glass, metal, and organic). Depending on the model's requirements, additional data points like sensor data (to detect material properties) and waste volume information may also be beneficial.


\subsection{MAS Development}
\label{sub:obj-devs}

The second sub-objective seeks to develop a multi-agent-powered model for waste sorting at the public refuse. Effectively mapping visual observations to optimal actions based on received rewards, a function approximator for waste image classification is one technique that will be utilized to achieve this goal. In other words, Convolutional Neural Networks (CNNs) enable the agent to "see" and better understand its environment, thereby enhancing decision-making in visually complex situations. Additional techniques include partially observable Markov Decision Processes and sequential decision-making algorithms. To achieve our goal, we will need to create a variety of individual models. This includes developing sophisticated sequential decision-making tools using reinforcement learning, as well as transformer-based computer vision models capable of identifying recyclable items in images, with an initial focus on bottles. We will organize and generate a dataset that consists of recyclable waste items, such as bottles, to address the object detection problem. Following that, we will construct multiple models for both segmentation and object detection. Our model development efforts will primarily focus on the transformer architecture-"Attention is All You Need" (Subakan et al., 2021).
In our investigation of scoped distributed and parallel algorithms within the context of sequential decision-making, particularly in reinforcement learning, Markov Decision Processes, and multi-armed bandits, we will define the waste sorting problem as a decentralized partially observable Markov decision process. This approach is necessary because our model involves multiple agents.


\subsection{Validation}
\label{sub:obj-valid}  

Thirdly, we intend to validate the developed model. Validating the model ensures that it generalizes well beyond the training data, helping to identify overfitting and underfitting. This process aligns the model's performance with business goals, builds confidence in its reliability, and enables early identification of potential issues for correction. We will evaluate different validation approaches, such as cross-validation, holdout validation, bootstrap methods, and domain-specific techniques, to determine their suitability for the specific purpose under consideration. Performance metrics such as accuracy, precision, and recall will be used for model validation. AI-driven models can be validated by following best practices, including using multiple evaluation metrics for a comprehensive assessment, simulating real-world conditions to test performance under likely scenarios, and implementing continuous monitoring to track performance over time. Additionally, it is important to align validation efforts with business objectives to ensure that metrics reflect specific goals, as well as to address bias and fairness to mitigate potential biases relevant to the scenario. Model validation is crucial, as it demonstrates how well our model performs with random test datasets. Our goal is to achieve a high level of validation accuracy.

% (fold)

